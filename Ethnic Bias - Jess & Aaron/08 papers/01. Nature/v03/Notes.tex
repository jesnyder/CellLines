Cell model breast cancer research has developed tool to study non-majority ethnicities in the patient population, with the notable exception of Native Americans. Reliance on a select subset of these tool as standards to study breast cancer perpetuated the bias. 


Figure 4 compares an ethnic group's presence in the breast cancer patient population to the group's representation in scientific publications. Unsurprisingly, the majority ethnic group, white patients, is also the most often cited cell line donor. This reflects the importance on comparison, emphasis on reproducibility, and the legacy of standards in the scientific community. However, has the cell model biased basic scientific research against the non-majority ethnicity? How well does the publication record map to the patient population? The majority ethnicity is over-represented in the publication record, compared to White's fraction of the patient population by 14\% in 1999, rising to 17\% in 2013. The fraction of Asians, Blacks, Hispanics, and American Indians cell line citation in the publication record has been less than their fraction of the patient population.  



The fraction of publication citing cell lines donated by white patients comprised 14\% more 

nd cell lines donated from white patients comprise a majority of the comprise a majority of the  grouped by CDC defined ethnicities 
Even thought white patient comprise a majority of the breast cancer patient population, cell line donated by white patient comprise more than 90\% of 
The fraction of publications citing cell lines donated by white patients is  was 14\% more than the fraction of white patients diagnosed with breast cancer, increasing since 1999. -17\%

 The next steps in breast cancer research characterizes the many sub-types of breast cancer as unique pathologies, relying on specific traits of the tumor and the patient's genetics to make decisions regarding treatment options. As the medical community pivots towards personalizing care based on genetic markers, we reflect on the inertia of breast cancer research. After acknowledging new genomic techniques change how we group patients, lets consider past use of ethnic groups. Before gene sequencing became widely available, the CDC grouped the US breast cancer patient population by ethnicity, to identify if and who was left out of improving health care options. The purpose being to expand healthy lifestyle choices, public awareness campaigns, diagnostics, and treatment options to include non-majority considerations. 

Transformative approaches to health care, such as personalized medicine, authenticate the importance of genetic variation on the effectiveness of treatment, physicians use genomic analysis to determine the course of treatment, circumventing the limitations of generalized treatments with tailored treatments. Here, we survey the genetic variation in breast cancer research with human cell models by characterizing the demographics of cell line donors represented in breast cancer research. 


